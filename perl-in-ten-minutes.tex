\documentclass{article}
\usepackage[utf8]{inputenc}
\usepackage{amsmath,amssymb,amsthm,pxfonts,listings,color}
\usepackage[colorlinks]{hyperref}
\definecolor{gray}{rgb}{0.6,0.6,0.6}

\usepackage{caption}
\DeclareCaptionFormat{listing}{\llap{\color{gray}#1\hspace{10pt}}\tt{}#3}
\captionsetup[lstlisting]{format=listing, singlelinecheck=false, margin=0pt, font={bf}}

\lstset{columns=fixed,basicstyle={\tt},numbers=left,firstnumber=auto,basewidth=0.5em,showstringspaces=false,numberstyle={\color{gray}\scriptsize}}

\newcommand{\Ref}[2]{\hyperref[#2]{#1 \ref*{#2}}}

% Shamelessly swiped from
% http://compgroups.net/comp.text.tex/using-ref-and-label-so-that-ref-points-to-the-l/245066
\makeatletter
\newcommand*{\Label}[2]{%
  \@bsphack
  \begingroup
    \label{#1-original}%
    \def\@currentlabel{#2}%
    \label{#1}%
  \endgroup
  \@esphack
}
\makeatother


\lstnewenvironment{asmcode}       {}{}
\lstnewenvironment{cppcode}       {\lstset{language=c++}}{}
\lstnewenvironment{javacode}      {\lstset{language=java}}{}
\lstnewenvironment{javascriptcode}{}{}
\lstnewenvironment{htmlcode}      {\lstset{language=html}}{}
\lstnewenvironment{perlcode}      {\lstset{language=perl}}{}
\lstnewenvironment{rubycode}      {\lstset{language=ruby}}{}
\lstnewenvironment{pythoncode}    {\lstset{language=python}}{}

\lstnewenvironment{resourcecode}{}{}

\title{Perl in Ten Minutes}
\author{Spencer Tipping}

\begin{document}
\maketitle
\tableofcontents

\section{Perl is not a good language}

Python, Ruby, and even Javascript were designed to be good languages -- and
just as importantly, to {\em feel} like good languages. Each embraces the
politically correct notion that values are objects by default, distances itself
from UNIX-as-a-ground-truth, and has a short history that it's willing to
revise or forget. These languages are convenient and inoffensive by principle
because that was the currency that made them viable. Perl is different. In
today's world it's a neo-noir character dropped into a Superman comic; but
that's only true because it changed our collective notion of what an accessible
scripting language should look like.

People often accuse Perl of having no design principles; it's ``line noise,''
pragmatic over consistent. This is superficially true, but at a deeper level
Perl is uncompromisingly principled in ways that most other languages aren't.
Perl isn't good; it's complicated, and if you don't know it yet, it will
probably change your idea of what a good language should be.\footnote{Along
these lines, I continue to insist that Perl 6 never happened and never will. In
the unlikely event that Perl 5 dies out to Perl 6 I'm jumping ship.}

\section{The sigil problem}

One of the first objections to Perl is silliness like this:

\begin{perlcode}
my $x = 5;              # $x is a scalar
my @x = (1, 2, 3);      # @x is an array (which has nothing to do with $x)
print $x[0];            # $x[0] is the first element of @x; why not @x[0]?
\end{perlcode}

Despite appearances, this is a great feature -- and in order to explain why, I
first need to describe how Perl looks at various types of data.

\subsection{Scalars, lists, and hashes}



\end{document}

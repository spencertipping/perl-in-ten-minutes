\documentclass{article}
\usepackage[utf8]{inputenc}
\usepackage{amsmath,amssymb,amsthm,pxfonts,color}
\usepackage[colorlinks]{hyperref}
\definecolor{gray}{rgb}{0.6,0.6,0.6}

\usepackage{caption}
\DeclareCaptionFormat{listing}{\llap{\color{gray}#1\hspace{10pt}}\tt{}#3}
\captionsetup[lstlisting]{format=listing, singlelinecheck=false, margin=0pt, font={bf}}

\newcommand{\Ref}[2]{\hyperref[#2]{#1 \ref*{#2}}}

% Shamelessly swiped from
% http://compgroups.net/comp.text.tex/using-ref-and-label-so-that-ref-points-to-the-l/245066
\makeatletter
\newcommand*{\Label}[2]{%
  \@bsphack
  \begingroup
    \label{#1-original}%
    \def\@currentlabel{#2}%
    \label{#1}%
  \endgroup
  \@esphack
}
\makeatother

\title{Perl in Ten Minutes}
\author{Spencer Tipping}

\begin{document}
\maketitle
\tableofcontents

\section{Perl is not a good language}

Python, Ruby, and even Javascript were designed to be good languages -- and
just as importantly, to {\em feel} like good languages. Each embraces the
politically correct notion that values are objects by default, distances itself
from UNIX-as-a-ground-truth, and has a short history that it's willing to
revise or forget. These languages are convenient and inoffensive by principle
because that was the currency that made them viable.

Perl is different.

In today's world it's a neo-noir character dropped into a Superman comic; but
that's only true because it changed our collective notion of what an accessible
scripting language should look like.\footnote{To get a sense of the historical
context it came from, consider that it and TCL were contemporaries and that
both Python and Ruby came late enough to inherit a strong OO influence from
Java. TCL was a brilliant language in its own way, but even more misunderstood
than Perl -- and it didn't go on to shape the future the way Perl did.} People
often accuse Perl of having no design principles; it's ``line noise,''
pragmatic over consistent. This is superficially true, but at a deeper level
Perl is uncompromisingly principled in ways that most other languages aren't.
Perl isn't good; it's complicated, and if you don't know it yet, it will
probably change your idea of what a good language should be.\footnote{Along
these lines, I continue to insist that Perl 6 never happened and never will. In
the unlikely event that Perl 5 dies out to Perl 6 I'm jumping ship.}

\section{\tt uncons}

Perl is one of the very few languages that gets this right, and, I think, the
only such mainstream language that isn't stack-based.\footnote{That is, it
supports applicative syntax -- though you can also use Perl as a stack language
if you're determined to. If you have no idea what any of this is, you should
check out Joy and FORTH (in that order unless you like assembly code) because
they will change your world.}

Here's the idea. We've got a {\tt cons} function and we want to write its
inverse, {\tt uncons}. How would we do it? Let's start in Ruby, where we can
get close.

\begin{verbatim}
def cons(h, t)
  Cons.new(h, t)
end
def uncons(c)
  [c.head, c.tail]
end

c = cons 3, 4
x, y = uncons c
\end{verbatim}

But {\tt uncons} isn't a true inverse:

\begin{verbatim}
c2 = cons(uncons(c))            # this dies
c2 = cons(*uncons(c))           # we have to do this instead
\end{verbatim}

And this illustrates a fundamental asymmetry in Ruby and in most applicative
languages:~a function can take multiple arguments, but can return only
one.\footnote{This is even true in most Scheme implementations because
arity-checking happens before CPS-conversion: \\
{\tt
(define uncons (lambda (c) (call/cc (lambda (k) (k (head c) (tail c)))))) \\
(define c2 (cons (uncons c)))   ; scheme dies here despite its validity in CPS
}}
If you think about how function calling works, of course, there's no
particularly good reason it needs to work this way:~functions are just
transformations against the call stack, and you could easily define a calling
convention that allowed a function to push multiple return values. And that's
exactly what Perl does.

\begin{verbatim}
sub cons {
  {head => $_[0], tail => $_[1]}
}
sub uncons {
  my ($c) = @_;
  ($$c{head}, $$c{tail});
}
$c = cons 3, 4;
($x, $y) = uncons $c;
$c2 = cons uncons $c;           # this works
\end{verbatim}

We have a problem, though:~suppose we actually want a function that returns an
array, not multiple values. Perl seems willing to blur the line between those
two concepts. In practice that's because in Perl there's no difference:~arrays
aren't things, they really are multiple values:

\begin{verbatim}
$c = cons 3, 4;
@returned = uncons $c;
$c2 = cons @returned;           # this works
\end{verbatim}

So Perl promotes arrays and hashes into language-level constructs because these
things have language-level semantics that their counterparts in Ruby and Python
don't. Having these semantics wouldn't be possible if Perl modeled these things
as objects.\footnote{The obvious question of arrays-within-arrays is handled by
{\em references}, which are single values that refer to complex data
structures. There's no syntactic ambiguity around them because they must be
dereferenced.}

\section{Sigils}

If \verb|@xs| is an array, you refer to its first element as \verb|$xs[0]|. To
explain why this works and why it's a feature, I first need to describe how
Perl looks at these two expressions; and before I get to that, I want to talk
about Perl's symbol table.

Perl lets you define multiple variables with the same name: \verb|@x| and
\verb|%x| are two completely different variables. Because they have the same
name, though, they're stored in the same location in the symbol table:~both are
accessible using \verb|${main::}{x}| (that is, the {\tt x} element of the {\tt
main::} hash). That hashtable value is a {\em typeglob}, which can be
dereferenced as a scalar, an array, a hash, or a function:

\begin{verbatim}
$xs = 5;
@xs = 1..10;
%xs = (foo => 1, bar => 2);
*xs = sub {"hi"};

my $v = ${main::}{xs};          # this is a typeglob
print $$v;                      # the scalar slot: prints 5
print @$v;                      # the array slot: prints 12345678910
print %$v;                      # the hash slot: prints foo1bar2
print &$v;                      # the code slot: prints hi
\end{verbatim}

(This insane detour has a purpose, I promise. Hang in there.)

Ok, now let's talk about what Perl looks at when it's figuring out which of
these typeglob entries you're working with. Let's say you've got a mystery
character \verb|?| that can be any of \verb|$|, \verb|@|, or \verb|%|, but Perl
doesn't know which. Writing \verb|?xs| is obviously ambiguous, but writing
\verb|?xs[0]| isn't; the reason is that square brackets only work on arrays.
\verb|?xs{foo}| also isn't ambiguous:~curly braces only work on hashes.

So Perl is in an interesting position:~it needs {\em some} leading character,
but it doesn't matter which one. Rather than having you repeat information Perl
already knows, it instead lets you specify something it doesn't know:~the
context of the thing inside the square or curly brackets.

\begin{verbatim}
$ perl -de1
perl> @xs = 'a'..'j';
perl> p @xs
abcdefghij
perl> p $xs[0]                  # pull one value: scalar context
a
perl> p @xs[1..5]               # pull multiple values: list context
bcdef
perl> p %xs[1..4]               # pull keys+values: list context
1b2c3d4e
\end{verbatim}

\end{document}
